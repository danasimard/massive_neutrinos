\documentclass{article}
\usepackage{graphicx}
\usepackage{amsmath}
\usepackage{fullpage}
\newcommand{ \wt }{\widetilde}
\usepackage{color}
\newcommand*{\mathcolor}{}
\def\mathcolor#1#{\mathcoloraux{#1}}
\newcommand*{\mathcoloraux}[3]{%
  \protect\leavevmode
  \begingroup
    \color#1{#2}#3%
  \endgroup
}

\usepackage{mdframed}
\newenvironment{aside}
  {\begin{mdframed}[style=0,%
      leftline=false,rightline=false,leftmargin=2em,rightmargin=2em,%
          innerleftmargin=0pt,innerrightmargin=0pt,linewidth=0.75pt,%
      skipabove=7pt,skipbelow=7pt]\small}
  {\end{mdframed}}

\begin{document}

\title{Fluid equations in comoving coordinates}
\author{Dana Simard}

\maketitle


Let us start with the equations in physical coordinates,
\begin{equation}\label{eqn:physcontinuity}
  \partial_t \rho + \partial_i (\rho u_i)  = 0 
\end{equation}
and 
\begin{equation}\label{eqn:physeuler}
   \partial_t (\rho u_i) +  \partial_j (\rho u_i u_j) = - \partial_i P - \rho \partial_i \Phi \;,
\end{equation}
where $\rho$ is the physical density, $u$ is the physical velocity, $P$ is the pressure and $\Phi$ is the physical gravitational potential.  Note that the indices $i$,$j$,$k$ are used for the spatial coordinates, while the time coordinate is written as $t$.  Also note we are summing over repeated indices but not worrying about whether indices are lowered or raised.  Finally the notation $\partial_i = \frac{ \partial }{\partial x_i} $ is used. 

Now, we want to change to super-comoving coordinates:\\
\begin{align*}
 \wt{\rho} &= \rho a^3\\
\wt{\phi} &= \phi a^2  \\
\wt{P} &= P a^5 \\
\wt{x_i} & = \frac{x_i}{a}  \\
\wt{v_i} &= v_i a  \\
 u_i &= v_i + \frac{\dot{a}}{a} x_i  \\
u_i &= a^{-1}\wt{v_i} + \dot{a}\wt{x_i }  \\
\end{align*}
where $\dot{a} = \frac{da}{dt}$. 
We will also change to super-conformal time, 
\begin{equation} 
  \frac{d}{dt} = \frac{1}{a^2} \frac{d}{d\wt{t}}\;.
\end{equation} 
This also changes the expressions for $u_i$,
\begin{equation}
  u_i = a^{-2} (\wt{v_I} + \frac{\dot{a}}{a} \wt{x_i} )\;,
\end{equation}
where now $\dot{a} = \frac{da}{d\wt{t}}$.
Due to the change in spatial coordinates we must also transform the derivatives:
\begin{align*}
\wt{\partial_i } &= a \partial_i \\
\partial_{t |x_i} &= a^{-2}\partial_{\wt{t} | \wt{x_i}} - a^{-2}\frac{\dot{a}}{a} \wt{x_i} \wt{\partial_i} \;.
\end{align*}

Since all variables are now in super-comoving and super-conformal form, I will drop the tidles from now on.

For the continuity equation, this transformation is as follows:
\begin{align*}
\frac{1}{a^2} \partial_t \frac{\rho}{a^3} - \frac{1}{a^2} \frac{\dot{a}}{a} x_i \partial_i \frac{\rho}{a^3} + \frac{1}{a} \partial_i ( \frac{\rho}{a^2} \frac{ v_i + \frac{\dot{a}}{a} x_i}{a} ) &= 0 \\
\frac{1}{a^5} \partial_t \rho - \frac{3}{a^5} \frac{\dot{a}}{a} \rho - \frac{1}{a^5}\frac{\dot{a}}{a} x_i \partial_i \rho + \frac{1}{a^5} \partial_i \rho v_i + \frac{1}{a^5} \frac{\dot{a}}{a} \partial_i \rho x_i &=0 \\
\partial_t \rho + \partial_i \rho v_i - 3 \frac{\dot{a}}{a} \rho - \frac{\dot{a}}{a} x_i \partial_i \rho + \frac{\dot{a}}{a} x_i \partial_i \rho + 3 \frac{\dot{a}}{a} \rho &= 0 \\
\partial_t \rho + \partial_i \rho v_i &= 0
\end{align*}

For the Euler equation, this transformation is:
\begin{align*}
  &\frac{1}{a^2} \partial_t ( \frac{\rho}{a^3} \frac{v_i +\frac{\dot{a}}{a} x_i}{a} ) 
- \frac{1}{a^2} \frac{\dot{a}}{a} x_j \partial_j ( \frac{\rho}{a^3} \frac{ v_i + \frac{\dot{a}}{a} x_i }{a} ) 
+ \frac{1}{a} \partial_j ( \frac{\rho}{a^3} \frac{v_i + \frac{\dot{a}}{a} x_i}{a} \frac{v_j + \frac{\dot{a}}{a} x_j }{a} ) 
= - \frac{1}{a^6} \partial_i P - \frac{\rho}{a^6} \partial_i \Phi \\
&\partial_t \rho v_i 
-4 \frac{\dot{a}}{a} \rho v_i 
+ \frac{\dot{a}}{a} x_i \partial_t \rho 
+ a^4 \rho \partial_t ( \frac{1}{a^4} \frac{\dot{a}}{a} x_i  ) 
- \frac{\dot{a}}{a} x_j  \partial_j \rho v_i 
- (\frac{\dot{a}}{a})^2 x_j \partial_j \rho x_i 
+ \partial_j \rho v_i v_j 
+ \frac{\dot{a}}{a} \partial_j \rho v_i x_j \\
&\;\;\;\;\;+ \frac{\dot{a}}{a} \partial_j \rho v_j x_i 
+ (\frac{\dot{a}}{a} )^2 \partial_j \rho x_i x_j 
= - \partial_i P - \rho \partial_i \Phi\\
&\partial_t \rho v_i + \partial_j \rho v_i v_j - 4 \rho v_i \frac{\dot{a}}{a} + \frac{\dot{a}}{a} x_i \partial_t \rho + \rho \partial_t ( \frac{\dot{a}}{a} x_i ) - 4 \rho (\frac{\dot{a}}{a})^2 x_i - \frac{\dot{a}}{a} x_j \partial_j \rho v_i - ( \frac{\dot{a}}{a} )^2 x_j \partial_j \rho x_i +\\ &\;\;\;\;\;\frac{\dot{a}}{a} x_j \partial_j \rho v_i + 3 \frac{\dot{a}}{a} \rho v_i + \frac{\dot{a}}{a} x_i \partial_i \rho v_j + \frac{\dot{a}}{a} \rho v_i + 3 ( \frac{\dot{a}}{a} )^2 \rho x_i + (\frac{\dot{a}}{a})^2 x_j \partial_j \rho x_i = - \partial_i P -\rho \partial_j \Phi \;.
\end{align*}
Using the continuity equation, this reduces to:
\begin{equation}
\partial_t \rho v_i + \partial_j \rho v_i v_j + \rho \partial_t \bigg(\frac{\dot{a}}{a} x_i \bigg) - \bigg( \frac{\dot{a}}{a} \bigg)^2 \rho x_i = - \partial_i P - \rho \partial_i \Phi \;.
\end{equation}

Now, we can decompose the potential $\Phi$ into the potential due to the homogeneous solution and the potential due to the density perturbation:
\begin{equation}
  \Phi = -\frac{\ddot{a}a}{2} \wt{x_i}\wt{x_i} + \phi
\end{equation}
where once again $\wt{x_i}$ indicates that we are using comoving spatial coordinates, but everything else is in physical coordinates.  So in the super-comoving and super-conformal coordinates:
\begin{align*} 
  \wt{\Phi} = a^{2}\Phi &=- a^{2} \bigg( \frac{a}{2}\frac{1}{a^3} \bigg( \frac{\ddot{a}}{a}  - 2 \bigg( \frac{\dot{a}}{a}\bigg)^2\bigg)  \wt{x_i}\wt{x_i}\bigg) + \wt{\phi} \\
  \wt{\Phi} &= -\frac{1}{2} \frac{\ddot{a}}{a}\wt{x_i}^2 +\bigg(\frac{\dot{a}}{a}\bigg)^2\wt{x_i}^2 +\wt{\phi} 
\end{align*}
I will again drop the distinction between the physical and comoving coordinates, and you may assume all coordinates are comoving.

We actually want the gradient of the potential:
\begin{align*}
  \partial_i \Phi &= \partial_i \bigg( -\frac{1}{2} \frac{\ddot{a}}{a}x_j^2 +\bigg(\frac{\dot{a}}{a}\bigg)^2x_j^2 +\phi \bigg)\\
  \partial_i \Phi &= -\frac{\ddot{a}}{a} x_i + 2 \bigg(\frac{\dot{a}}{a}  \bigg)^2 x_i + \partial_i \phi
\end{align*}
We also need the term:
\begin{align*}
 \wt{ \partial_t}\bigg( \frac{1}{a}\frac{da}{d\wt{t}} \wt{x_i} \bigg) &= a^2 \partial_t \bigg( \frac{1}{a} a^2 \frac{da}{dt} \frac{x_i}{a} \bigg)\\
&=a^2 x_i \frac{d^2a}{dt^2} \\
&= a^2 a \wt{x_i} \frac{1}{a^2} \frac{d}{d\wt{t}}\bigg( \frac{1}{a^2} \frac{da}{d\wt{t}} \bigg) \\
&= \frac{\ddot{a}}{a} \wt{x_i} - 2 \bigg( \frac{\dot{a}}{a} \bigg)^2 \wt{x_i} 
\end{align*}


Lets put this into the Euler equation and we find:
\begin{align*}
\partial_t \rho v_i + \partial_j \rho v_i v_j + \rho
  \frac{\ddot{a}}{a} \wt{x_i} - 2 \rho \bigg( \frac{\dot{a}}{a}
  \bigg)^2 \wt{x_i}  - \bigg( \frac{\dot{a}}{a} \bigg)^2 \rho x_i 
 &= - \partial_i P +\rho\frac{\ddot{a}}{a} x_i  -
   2\rho\bigg(\frac{\dot{a}}{a}\bigg)^2x_i-\rho \partial_i \phi\\
\partial_t \rho v_i + \partial_j \rho v_i v_j &= -\partial_i P -
                                                \rho \partial_i \phi
\end{align*}

\begin{aside}
What happens if we set $a=1$? Then $\dot{a}=0$, $H=0$, and all of the
super comoving variables are equal to the physical variables.  In this
case, the Euler equation above remains unchanged, and the original
Euler equation becomes:
\begin{align*}
  \partial_t \rho u_i + \partial_j \rho u_i u_j &= - \partial_i P -
  \rho \partial_i \Phi \\
\wt{\partial_t} \wt{\rho} \wt{v_i} + \wt{\partial_j} \wt{\rho} \wt{v_i} \wt{v_j} &= - \wt{\partial_i} \wt{P} - \wt{\rho} \wt{\partial_i} \wt{\phi}
\end{align*}
as expected, since the Hubble flow in this case is zero and only the
perturbations remain. This matches the Euler equation above as expected.
\end{aside}

% Now, in order to get this in the correct form, we see that a specific relation must hold:
% \begin{equation}
%   \partial_t \bigg(\frac{\dot{a}}{a} x_i \bigg) = \bigg(\frac{\dot{a}}{a} \bigg)^2 x_i 
% \end{equation}
% But this seems to only hold if H is constant.  In physical coordinates, this condition is:
% \begin{align*}
%  a^2 \partial_t ( \dot{a} x_i ) - a \dot{a}^2 x_i &= 0 \\
% x_i  \ddot{a}  + \dot{a} \partial_t x_i - \frac{ \dot{a}^2}{a} x_i &= 0 \\ 
% \frac{\ddot{a}}{a} - \frac{ \dot{a}^2 }{a^2} &= 0 \\
% H^2 + \dot{H} - H^2 &= 0 \\
% \dot{H} &= 0 \\
% \end{align*}
% Assuming that the time derivative the $x_i$ is zero (is this true?) and using
% \begin{equation}
%   \ddot{a} = \partial_t \dot{a} = \partial_t aH = \dot{a}H + a \dot{H} \;.
% \end{equation}

% So we see that under the assumption that $\dot{H} = 0 $, the Euler equation can still be written in a conservation form in the superconformal and supercomoving coordinates:
% \begin{equation}
%   \partial_t \rho v_i + \partial_j \rho v_i v_j = - \partial_i P - \rho \partial_i \Phi
% \end{equation}

Now we want to obtain one equation from the Euler and continuity equations in these coordinates.  Start by taking the time derivative of the continuity equation:
\begin{equation}
  \partial_t^2 \rho + \partial_t \partial_i \rho v_i = 0 
\end{equation}

Taking the divergence of the Euler equation gives:
\begin{equation}
  \partial_i \partial_t \rho v_i + \partial_i \partial_j  \rho v_i v_j
      = - \partial_i^2 P - \partial_i \rho \partial_i \phi 
\end{equation}

We can combine these two equations to get the single equation:
\begin{equation} 
  \partial_t^2 \rho - \partial_i \partial_j \rho v_i v_j =  \partial_i^2 P + \partial_i \rho \partial_i \phi
\end{equation}

Now we want to convert back into physical coordinates for $\rho$, $v$, $P$, and $\Phi$ although we will keep the conformal $x$ and the supercomoving $t$ coordinates (sorry for the continual change of notation).   We will also write the density in terms of the average density plus a perturbation, so that $\wt{\rho} = a^3 \rho_0 (1+\delta)$, $\wt{v_i} = a v_i$, $\wt{P} = a^5 P$ and $\wt{\Phi} = a^2 \Phi$.  This gives:
\begin{equation}
\partial_t^2 \bigg(a^3 \rho_0 (1+\delta) \bigg)
  - \partial_i \partial_j \bigg(a^3 \rho_0 (1+\delta) a^2 v_i
  v_j\bigg)  =  a^5 \partial_i^2 P + a^5 \partial_i \bigg(\rho_0 (1+\delta) \partial_i \phi \bigg)\;.
\end{equation}
We can write the first term as:
\begin{align*}
  \partial_t \partial_t ( a^3 \rho_0 (1+\delta) ) &= \partial_t( a^3 \rho \dot{\delta} + 3 \frac{\dot{a}}{a} a^3 \rho_0 (1+\delta) - 3 \frac{\dot{a}}{a^3} \rho_0 (1+\delta)) \\
&= \partial_t( a^3 \rho_0 \dot{\delta})\\
&= a^3 \rho_0 \ddot{\delta} - 3 a^3 \frac{\dot{a}}{a} \rho_0 \dot{\delta} + \frac{\dot{a}}{a} \rho_0 \dot{\delta} \\
&= a^3 \rho_0 \ddot{\delta} 
\end{align*}
where we have used the continuity equation to determine that the homogenous solution density $\rho_0$ scales as $a^3$.

Now we may write the entire equation as:
\begin{equation}
  \ddot{\delta} - a^2 \partial_i \partial_j \bigg((1+\delta) v_i v_j\bigg) =  a^2 \frac{\partial_i^2 P}{\rho_0} + a^2  \partial_i \bigg((1+\delta) \partial_i \phi \bigg)
\end{equation}

We can break this into the first, second and third order terms (in the
perturbations):
\begin{equation}
  \ddot{\delta}  \mathcolor{blue}{- a^2 \partial_i \partial_j \bigg(
    v_i v_j \bigg) }  \mathcolor{red}{-a^2 \partial_i \partial_j
    \bigg( \delta v_i v_j \bigg) } = a^2 \frac{\partial_i^2 P
  }{\rho_0}  + a^2 \partial_i^2 \phi  \mathcolor{blue}{+
    a^2 \partial_i \bigg( \delta \partial_i \phi \bigg) } 
\end{equation}
where the first order terms are black, the second order terms are
blue, and the third order terms are red.

If we drop the third order terms and expand the second term, we end
up with:
\begin{equation}
  \ddot{\delta} - a^2 \nabla \cdot ( \mathbf{v} ( \nabla \cdot
  \mathbf{v} ) + ( \mathbf{v} \cdot \nabla ) \mathbf{v} ) - a^2
  \frac{\nabla^2 P}{\rho_0} = a^2 \nabla^2 \phi + a^2 \nabla \cdot ( \delta
  \nabla \phi) 
\end{equation}

What do we do with the third term? In k-space this becomes:
\begin{align*}
  \nabla \cdot ( \mathbf{v} (\mathbf{v} \cdot \nabla ) + ( \mathbf{v}
  \cdot \nabla ) \mathbf{v} ) &= \partial_i ( v_j \partial_j v_i +
                                v_i \partial_j v_j ) \\
  &= \partial_i v_j \partial_j v_i + v_j \partial_i \partial_j v_i
    + \partial_i v_i \partial_j v_j + v_i \partial_i \partial_j v_j \\
  &= k_i v_j \ast k_j v_i + v_j \ast k_i k_j v_i + k_i v_i \ast k_j
    v_j + v_i \ast k_i k_j v_j \\
  &= k_i v_j \ast k_j v_i + 2 v_j \ast k_i k_j v_i + k_i v_i \ast k_j
    v_j  \\
  &\equiv A(\mathbf{v}(\mathbf{k}),\mathbf{k})
\end{align*} 

So in k-space we have:
\begin{equation}
  \ddot{\delta} - a^2 A(\mathbf{v}(\mathbf{k}),\mathbf{k}) - a^2
  \frac{\nabla^2 P }{\rho_0} = a^2 \nabla^2 \phi + a^2 \nabla \cdot(
  \delta \nabla \phi)
\end{equation}

From the Poisson equation, we have, in the matter dominated region:
\begin{equation}
  \nabla^2 \phi = 4 \pi G a^2 \bar{\rho}\delta = \frac{3}{2} H^2
  a^2 \delta 
\end{equation}
In Fourier space, this means that we can write the potential:
\begin{equation}
  \phi = \frac{3}{2} \frac{H^2(t)a^2}{k^2} \delta = \frac{3}{2}
  \frac{H^2(\tau) }{k^2 a^2} \delta 
\end{equation}
where $t$ is the physical time and $\tau$ is the super-conformal
time.  We can then write the full equation:
\begin{equation}
  \ddot{\delta} - a^2 A(\mathbf{v}(\mathbf{k}),\mathbf{k}) - a^2
  \frac{\nabla^2 P }{\rho_0} = \frac{3}{2} \frac{H^2(\tau)}{k^2} \nabla \cdot((1+
  \delta) \nabla \delta_c)
\end{equation}
So with must calculate $A(\mathbf{v}(\mathbf{k}),\mathbf{k})$ in order
to do this integral.

\end{document}